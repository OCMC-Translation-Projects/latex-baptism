% Author: Michael Colburn
% Purpose: controls various options for typesetting liturgical text

%==================================================
% Show keys?
%
% Controls whether the PDF will display the key used to retrieve the
% text.
%
%\def\showKeys{} % comment out to NOT display keys
% 
%==================================================

%==================================================
% Actors on separate line?
%
% Controls whether the actor will appear on its on line.
% For example, 
%
% If enabled:
% Priest: In the name of the Father and of the Son...
%
% If not enabled:
% Priest
% In the name of the Father and of the Son...
%
%\def\actorOnSeparateLine{} % comment out to NOT display keys
% 
%==================================================

%==================================================
% One language or two?
% 
% Controls whether we will have one language (monolingual)
% or two languages (bilingual)
% Bilingual text is typeset as two columns on each page.
%
%\newcommand*{\createbilingual}{} % comment out to do monolingual
%
%==================================================

%==================================================
% the name of the directory contain the language tex files
\def\ltResourceDir{resources/}
% 
%==================================================

%==================================================
% languages
\def\ltLeftMain{"en_UK_lash"}
\def\ltLeftAlt{"en_KE_oak"}
\def\ltLeftLextionaryMain{"en_US_rsv"}
\def\ltLeftLextionaryAlt{"en_US_saas"}
% 
%==================================================

%==================================================
% Which language or languages?
% 
% Controls language or languages to use.
% Each language must have a folder with that name.
% If you are typesetting just one language (monolingual)
% then comment out the \def\langRight
%
\def\langLeft{\ltLeftMain} % if monolingual, only this language will be used
\def\langRight{"gr_gr_cog"} 
%
%==================================================
